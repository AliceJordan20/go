\chapter{Introduction}
The Universe can be described by a small number of fundamental particles that make up all of matter. These building blocks of matter are governed by four fundamental forces; gravity, electromagnetic, strong and weak. These particles can be broken down into two elementary groups; quarks and leptons collectively known as  \textit{fermions}, where a fermion is defined as having a half-integer spin . The way fundamental forces interact with matter is through force carriers, or \textit{vector bosons}, vector bosons have integer spins. A total of eight gluons acts as the bosons for the strong force, the W\textsuperscript{-}, W\textsuperscript{+} and Z\textsuperscript{0} are all bosons for the weak force, the photon is the boson for the electromagnetic force and finally the yet undiscovered theoretical graviton is the boson for gravity. A final addition, is the much anticipated and recently discovered Higgs boson which is known as a \textit{scalar boson} due to having a spin of 0\cite{Braibant:2012vky}, more to come on this later. All of this fits into the Standard Model which sets out and describes how these elementary particles and forces dictate physical processes and phenomena. 

The discovery of the Higgs came on the 4\textsuperscript{th} July 2012 when the ATLAS and CMS experiments announced that a particle had been discovered at the Large Hadron Collider with a mass of 126GeV which was consistent with that predicted by the Standard Model\cite{Higgs:1998491}. This discovery resulted in the Nobel prize being awarded to Peter W. Higgs François Englert in 2013 for the theoretical discovery of the BEH mechanism.  Even today, 5 years later, very little is known about the Higgs, especially how it couples to other particles. Briefly put, interacting with the Higgs  is thought to be how particles have mass, so the Atlas experiment is currently looking into how the Higgs couples to certain particles. The prominent decay mode of the Higgs boson is to a bottom/anti-bottom quark pair, H$\,\to\,b\overline{b}$, it has the largest branching ratio of ~58\%\cite{Djouadi:1997yw}. This is of particular interest as the bottom quark has such a disproportionately large mass compared to the other quarks disregarding the top quark, but it is unkwnown as to why this is. It is hoped that through  studying the coupling of particles such as the bottom quark to the Higgs that this mass discrepancy can be understood along with an explanation as to why particles have the specific mass they do.

Despite the large branching ratio of the H$\,\to\,b\overline{b}$, there is an overwhelming amount of background processes that swamp the signal Higgs events when produced at the LHC. This provides the motivation for this report; to create a machine learning algorithm that is able to successfully distinguish between signal and background events. This process could then be replicated on LHC data to aid ATLAS and other experiments in their quest to further understand the coupling of the Higgs. 

This was done by the generation of pseudo data using a Monte Carlo generator and then from these data sets extracting a number of distinguishing variables between the signal and background events, such as energy, transverse momentum. Then by feeding these variables into a Multivariate Analysis program (MVA), a classifier was produced that should give a distinction between signal and background events and when applied to unknown data, be able to sort the samples into signal or background. 




